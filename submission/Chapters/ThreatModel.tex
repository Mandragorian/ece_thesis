\chapter{Threat Model}
\label{chapter:threat_model}

\newcommand{\secadv}{$\mathcal{O}$ }
\newcommand{\privadv}{$\mathcal{M}$ }
\newcommand{\conadv}{$\mathcal{T}$ }

In \cite{mpotr} a threat model was specified. First, three adversaries are introduced. Then, we describe the goals of the Protocol regarding each adversary.

\section{The Adversaries}

\subsection{Security adversary $\mathcal{O}$}
This adversary's goal is to read the messages of the chatroom.
Let $T_c^{\hat{X}}$ be the transcript of chatroom $c$ owned by participant $\hat{X}$.
Then, $\mathcal{O}$ is successful, if he can read any message from transcript $T_c^{\hat{A}}$, where $\hat{A}$ is an honest participant, without receiving it from transcript $T_c^{\hat{X}}$ for any participant $\hat{X}$ where $\hat{X} \ne \hat{A}$.
While $\mathcal{O}$ can, both passively and actively, control the network, decrypt messages sent in other chatrooms and even participate in other sessions, he has limited access on the room he wants to attack.
Not only can he not participate in the session under attack, but he also cannot ask for any secret shared between the participants of the specific room.
He has the ability to inject messages of his liking in the chatroom by asking an, otherwise honest, user.
In essence $\mathcal{O}$ is a somewhat formal definition of the notion of IND-CPA attacks in the multiparty setting.

\subsection{Privacy adversary $\mathcal{M}$}

The privacy adversary aims to break the deniability of the protocol.
He is successful if he can prove to a judge $\mathcal{J}$ that a user $\hat{A}$ participated in, read messages from or authored messages in chatroom $c$.

His restrictions are very few.
He can collaborate with $\mathcal{J}$ before the creation of $c$, participate fully in $c$ and even force $\hat{A}$ to reveal his long term secrets in front of the judge.

\subsection{Consensus adversary $\mathcal{T}$}

\subsubsection{Definition of Consensus}

For two participants $\hat{A}$ and $\hat{B}$, consensus is reached on $T_{C_1}^{\hat{A}}$ when $\hat{A}$ believes $\hat{B}$ claims to have a transcript $T_{C_2}^{\hat{B}}$ such that:

\vbox{%
\begin{itemize}
  \label{consensus_def}
  \item $C_1$ has the same set of participants as $C_2$;
  \item $C_1$ and $C_2$ are the same chat room instance;
  \item $T_{C_2}^{\hat{B}}$ has the same collection of messages as $T_{C_1}^{\hat{A}}$;
  \item $T_{C_2}^{\hat{B}}$ and $T_{C_1}^{\hat{A}}$  agree an each message's origin.
\end{itemize}
}

Notice that the above definition is not symmetric.
This means that $\hat{A}$ can reach consensus with $\hat{B}$ without necessarily $\hat{B}$ reaching consensus with $\hat{A}$.

The interpretation of the term "collection of messages" is intentionally left unclear.
This way, each application can handle the ordering of the messages in different ways.

\subsubsection{$\mathcal{T}$'s goal}

$\mathcal{T}$ is successful when he is able to force an honest user $\hat{A}$ to believe that consensus is reached with another honest user $\hat{B}$ when at least one condition from \ref{consensus_def} does not hold.
Notice that only $\hat{A}$ and $\hat{B}$ must be honest.
$\mathcal{T}$ can otherwise control other users as he sees fit.

\section{The Goals of the Protocol}

The protocol must provide some defence mechanisms against all of the above.

The security adversary \secadv can in no way be successful.
The protocol must ensure that no one outside a chat room can read messages authored for it.
This would be a catastrophic failure.

To defend against the privacy adversary \privadv is also crucial.
One might think, that since an attacker can not read messages, it won't make much difference if they are not deniable.
However there are many situations that even evidence that you talked to someone can be incriminating.

If, for example, Ed wanted to reveal to a journalist some evidence about the wrongdoings of a state organization and did that in a non deniable manner then he would be busted.
The correlation between him contacting the journalist and the subsequent release of the information he revealed would mark him as a whistle blower.

Lastly, the consensus adversary \conadv is not strictly cryptographic but is of no less importance.
If an attacker could manipulate the transcripts of two different users, but then convince them that they have received the same messages he could have quite some power over them.
Every multi-party chatting protocol must ensure that all participants view the same messages\footnote{The order of the messages is also very important but this is an open problem and, while various solutions have been proposed, it is not addressed by this work. See \ref{sections:message_ordering} for a theoretic approach and \ref{sections:ordering_with_oldblue} for our proposed solution.}.
