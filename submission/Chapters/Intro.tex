\chapter{Introduction}

\label{chapter:introduction}

\newcommand{\dhname}{Diffie--Hellman}
\newcommand{\tdhname}{Triple Diffie--Hellman}


%--------------------------------------------

\section{Summary}

Not much time have passed since Edward Snowden revealed the plans that a certain intelligence agency has for the internet.
While the world has always suspected that the various 3 letter agencies had the capability of controlling the network at a large scale, everybody was shocked with the confirmation of those suspicions.

This means that in a world where the need for easy instant communication must overcome the threat of constant surveillance, end-to-end encryption has become a necessity.
Thankfully this problem is solved in the case of a two party setting, where only 2 participants are exchanging messages, as a number of applications like OTR in pidgin and adium, Signal app for mobiles etc. provide end-to-end encryption in a user friendly manner.
However, multi-party chat rooms are also very prominent in everyday communications and here things are somewhat trickier.
On one hand, security in multi-party chat rooms is still an open problem theoretically.
On the other, the existing implementations are either heavily mobile-centric or they are not licensed under a free software license.
Additionally they all implement the same protocol, called Signal.
And while Signal is a powerful protocol which provides strong encryption, we believe that more solutions should be implemented and tested, since, as we said, secure multi-party chat rooms are still an open problem.

In this thesis we take a different approach from the signal protocol.
We base our work on the Multi-party Off-the-Record messaging protocol ( from now mpOTR ) paper by I. Goldberg et al. \cite{mpotr}.
Our contribution is both theoretical and practical.
From a theoretic perspective, the mpOTR paper was not complete, as, as we will see later, two major building blocks, a Deniable Signature Key Exchange (DSKE) and a Group Key Agreement, were treated as black boxes.
From a practical perspective we implemented the protocol almost from nothing. We had to engineer our implementation in such a way that its security is easily reviewable, and, at the same time, facilitates the free software development model, where contributions in the source code are made from several independent authors.

As a result of our work's duality we will take a look in our contributions from the point of view of a cryptographer and the point of view of a software engineer.

\subsection{Cryptographer's PoV}

In this section we will discuss very briefly the theoretical focal points of our work.

As already stated the protocol proposed in \cite{mpotr} left out two major gaps.

\subsection{Software Engineer's PoV}
