\chapter{Related Work}
\label{chapter:related_work}

As we have already stated, end-to-end encrypted instant messaging is a very trending topic in  the crypto community.
Consequently, a plethora of protocols and implementations achieving the above goal exist, the majority of them handling the two-party case.

Here we will present a brief collection of the aforementioned protocols.
Our goal is not to be exhaustive, but rather to give credit to the authors of those protocols.
Either because their ideas gave us inspiration, or because we experienced first hand the difficulties of implementing and designing secure communication applications, and would like to acknowledge their contributions.

\section{Two-party Protocols}

First we will take a look at protocols providing conversations between two parties.

\subsection{Off-the-Record Messaging}

This protocol could not be missing from this section.
It is designed by the same team which authored the original Multi-part OTR paper \cite{mpotr}.

As far as we know it is the first complete protocol to provide end-to-end encrypted Instant Messaging with strong cryptography.
It set the ground for many other protocols and algorithms, like the axolotl key ratchet, which later evolved into the Signal protocol.

\subsection{Vuvuzela}

\noindent\enquote{\itshape We kill people based on metadata}\bigbreak

\hfill {\small Michael Hayden, former NSA director}
\bigbreak

While end-to-end encryption is an essential component for any privacy protecting protocol, it is not perfect on its own.
The metadata can reveal a bunch of crucial information that can be used to even determine the contents of encrypted data.

Vuvuzela \cite{vuvuzela} is a protocol that provides strong metadata privacy that scales.
It utilizes onion encryption (like TOR) to hide metadata.
Messages are sent in rounds, and noise packets are injected in the network in order to defend against traffic analysis attacks.

\section{Multi-party Protocols}

And now let's see some multi-party protocols.

\subsection{Flute}

Flute is a secure multiparty messaging protocol, currently available as a weechat plugin for irc chatrooms.
It provides end-to-end encryption, but does not offer other desirable properties like deniability or chatroom message consistency.

Flute does not provide everything that is needed for a multi-party protocol to be completely secure.
It's simplicity however, allowed it to have a quick implementation, and also provides the challenge to figure out what protocol are crucial in multi-party messaging, and must be added, or do not really enhance the security and should better be left out.

\subsection{Signal}

Signal, developed by Open Whisper Systems, is currently the most complete solution to the multi-party messaging problem.
It is a robust protocol providing all the necessary properties like \emph{confidentiality, authentication, deniability} and \emph{forward secrecy}.
It does not provide \emph{transcript consistency} but the double-ratchet it uses provides some resistance against reordering attacks.

It is available as an Android and iOS application, and for two party conversation can be used through the chrome browser.
Independently from its authors various applications also use it.
What'sApp, Viber, and Facebook Messenger to name a few, are some of the applications that use the Signal protocol to provide secure chats.
Cryptocat is a firefox plugin that closely follows the Signal approach and is completely open source.
