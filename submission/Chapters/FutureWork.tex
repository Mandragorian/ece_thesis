\chapter{Future Work}
\label{chapters:FutureWork}

\section{Message Fragmentation}
Some networks may have a message length limitation that is too small to contain an mpOTR message. To solve this problem, OTR has utilized a fragmentation mechanism since version 2.

The OTR fragmentation mechanism is intuitive. On the sender's side, messages are splitted into a sufficient number of pieces N, so that every fragment does not exceed the specified length limit. Each fragment contains a sequence number k, the value of N and the actual piece. The number k indicates the position of the piece in the whole message, starting from 1 and ending to N. On the recipient's side, the fragments are accumulated so that after receiving the piece with its sequence number k equal to N the whole message can be reconstructed.

This fragmentation mechanism works properly as long as the fragments are delivered in-order. In case of out-of-order delivery the algorithm implements a fragment forgetting strategy that finally rejects the message. A more severe problem arises when fragments from different messages are delivered intermixed. Since each fragment contains no information identifying the message it belongs to, the only distinctive information is the number N. In case N contained in a newly received fragment is different than the one contained in the so far accumulated fragments, the so far accumulated fragments are forgotten. But what happens if fragments from different messages splitted into the same number of pieces N are delivered in an intermixed order? Hopefully, the sequence numbers k won't form a valid sequence. In the extreme scenario the latter happens a non-valid message will be reconstructed.

The problems described above are actually unlikely to happen in a two-party communication context. But such a mechanism in a multi-party context would be a bad choice. Even if fragments are accumulated separately for each sender, there is a sporting chance that fragments from different messages sent during the setup procedure will be delivered intermixed. Rejecting such a message would require a restart of the whole setup procedure, and that could possibly occur indefinitely.

\section{Message consistency in constant space}
In \cite{mpotr} a straightforward approach is followed. In order to check if
each participant has received the same set of messages all the messages from
each user must be stored, and during the shutdown phase be lexicographically
ordered and hashed. While this achieves our purpose it requires $O(M)$ space
where $M$ is the number of messages.

We believe that the same effect can be achieved in constant space by using
cryptographic accumulators. One can find more about this primitive in \cite{accum_def}.

Since such accumulators are collision free but at the same time quasi-commutative,
they are ideal for our purposes. We can feed the accumulator with the incoming
messages in whichever order they arrive at each participant. The quasi-commutative
property guarantees that if two participants have received the same set of messages
then their accumulators will arrive at the same value in the end.

Thus, we have removed the need to store the messages in order to sort them during
the shutdown phase. We only need to store the value of the accumulator which,
of course, is constant.

\section{Message Ordering with OldBlue Protocol}
First let's see what we can and cannot do:
A multi-party chat room is a distributed environment.
This means that there is no central "authority" which can decide if a message came before another.

Also, since we are in a zero trust setting we cannot utilise any values over which the participants have full control, their local clocks for example.

The only information for which we must trust the other participants is what messages they have seen.
We have no other option on this one as we cannot know if an attacker has actually stopped messages from getting to them or if they are dishonest and lie to us.

