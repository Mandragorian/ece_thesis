\chapter{Theoretic Background}
\label{chapters:TheoreticBackground}

In this chapter we will present some basic theoretic background.
All the concepts, cryptographic primitives, and ideas presented here are essential building blocks of the protocol proposed in this thesis.
As a result before someone carries on forward in this document, she should first have a basic understanding of this chapter.

\section{Symmetric Encryption}

The idea of symmetric encryption is quite simple.
We suppose that there exists an algorithm called $E$, which takes two inputs.
One input is a secret, called the key, and the other is some data we would like to encrypt.

Another algorithm called $D$ again takes two inputs and is the reverse of $E$.
This means that if we call $D$ with inputs the output of $E$ under some key $k$, and $k$ itself, the output will be the plaintext data. This is show in the below equation:

\[
  m = D(k, E(k, m))
\]


The goal of any key exchange protocol is to allow two users to agree on a secret value, known only to them, by exchanging some publicly known information.
This might be counter intuitive at first but as we will see in the next section this goal can be achieved with a very simple construction.

Basically a key exchange protocol is any mechanism that provides two functions.
One of them returns a tuple containing some private and some public information.
The public information returned by that function must be sent to the other so that he can calculate the secret value.
The other accepts the public values of the other user and the private values of this user.
It returns the secret value.

\begin{figure}[H]
  \begin{align*}
    <priv, pub> &= genkey() \\
    secret &= calculate\_secret(priv_i, pub_j)
  \end{align*}
  \caption[The interface of a Key Exchange Protocol]{The KEP interface. $pub_j$ is the public information of user $j$ and $priv_i$ the private information of user $i$.}
\end{figure}

\subsection{\dhname key exchange}

The \dhname key exchange is, as the name suggests, a key exchange protocol.
It plays a major role in our proposed protocol, since it is a building block of many of its components.
In this section we will examine how this KEP is constructed, and what public values must be exchanged.

First, we remind to the readers the operation of multiplication modulo a number, $\oplus$.
We want to multiply two numbers $a$ and $b$ modulo a number $n$, called "the modulo".
This means that we first multiply the two numbers as usual.
Then we calculate the remainder of the result when it is divided by the modulo.
This remainder is the result of the multiplication of those two numbers $a$, $b$ modulo $n$.
This means that if:

\[
  ab = np + r
\]

Then:

\[
  a \oplus b  = r
\]
.

In this section we will also symbolize:

\[
  a^x \equiv a \oplus \dots \oplus a = a^x \mod p
\]

These are all the maths needed to understand how the \dhname key exchange works.
To understand why it is also secure is a whole different matter and we will not cover it in this publication.

The \dhname construction supposes that the two users already agree on two values $g$ and $p$ which are publicly known.
The number $p$ must be prime and is called "the modulo" of the protocol.
The number $g$ is called the "generator" and has the property that for every number $k \in [1 \dots p-1]$ there exists a number $l$ in the same range such that $g^l = k$.

The private information for a user is any random integer $x$ such that $ 1 < x < p$.
The public information is $g^x \mod p$ (any exponentiation from now on will be modulo $p$).

The calculation of the shared secret is trivial.
A user $i$, with private information $x$, and public $g^x$, receives the public information $g^y$ of another user $j$.
He then calculates the value $s = (g^y)^x$ which is the shared secret.
Now note that with $i$'s public information, $j$ can also calculate the same value $s = (g^x)^y$.
The calculated value is the same for the two users since:

\[
  (g^y)^x = (g^x)^y = g^{xy}
\]

In algorithm \ref{algo:dh_genkey} we see the $genkey$ function, and in algorithm \ref{algo:dh_calculate_secret} the $calculate\_secret$ function.

From now on, the public and private information of a user will be called public and private keys accordingly.

\begin{algorithm}[h]
  \KwResult{The private and public information needed by the protocol}
  \Begin{
    x := $random_in_range(1,p-1)$

    p := $g^x$

    \Return{$<x, p>$}
  }
  \caption{The $genkey$ function}
  \label{algo:dh_genkey}
\end{algorithm}

\begin{algorithm}[h]
  \KwIn{$g^y$: the public value of the other user, $x$: the secret value of the user calling the function}
  \KwResult{The shared secret, which is known only to the two users}
  \Begin{
    \Return{$(g^y)^x$}
  }
  \caption{The $genkey$ function}
  \label{algo:dh_calculate_secret}
\end{algorithm}

\section{Man-in-the-Middle attacks}

\dhname is a great protocol for calculating shared secrets it has a grave disadvantage.
Consider the following scenario, where Eve, an evil attacker can control the network so that she can read, drop, and inject packets:

User $i$ wants to privately communicate with user $j$.
She generates a private key $x$ and sends the public key $g^x$ to $j$.

Eve interjects and copies and then drops the packet containing $i$'s public key.
She generates a private key $x\prime$ and the public counterpart $g^{x\prime}$.
She performs a \dhname exchange with $i$'s key and calculates $s_1 = g^{xx\prime}$, since she knows $x\prime$.
She then sends her public key to user $j$.

$j$ believes that the public key he just received belongs to $i$.
He generates his private key $y$, and sends the public key $g^y$ to $i$.
He also calculates $s_2 = g^{yx\prime}$ what he thinks is a shared secret known only to him and $i$.

Eve interjects again to copy and drop the just sent package from the network.
She calculates $s_2 = g^{yx\prime}$, again she knows $x\prime$.
Then she sends her public key $g^{x\prime}$ to $i$, posing as $j$.

Now $i$ receives a public key which she thinks belongs to $j$.
Like $j$ she calculates the shared secret $s_1 = g^{xx\prime}$ which she thinks is only known between her and $j$.

The situation now is that $i$ and $j$ use two different secrets to communicate which are both known to Eve.
Eve can decrypt any message sent by $i$, for example, since she knows $s_i$.
She then can re-encrypt it using $s_2$ and relay it to $j$.

Nor $i$ neither $j$ will be able to know that such tampering is taking place and will continue to communicate, thinking that everything is fine.

The above scenario was described with the \dhname key exchange in mind.
However any similar situation where a malicious attacker can get between two communicating partners who think they are talking directly to each other is known as a Man-in-the-Middle attack.

\section{\tdhname key exchange}

\tdhname is another key exchange protocol.
It is an improvement of \dhname which introduces negligible overhead in the complexity of the protocol.
With doubling length of the first message containing the public keys of a user, this protocol provides both authentication and deniability.
